% ************************** Thesis Abstract *****************************
% Use `abstract' as an option in the document class to print only the titlepage and the abstract.
\begin{abstract}
  This thesis investigates Bach's music composition style using deep sequence
  learning. We develop \textit{BachBot}: a deep LSTM model for automatic
  stylistic composition of polyphonic music in the style of Bach's chorales.
  Our approach encodes music scores into a sequential format, reducing the task
  to one of sequence modeling. Traditional $N$-gram language models are found
  to be insufficient, prompting the use of RNN sequence models. We find a
  $3$-layer stacked LSTM performs best and conduct analyses and evaluations to
  understand its success and failure modes.

  Unlike many previous works where model architecture is carefully designed
  using an understanding of music theory, we \emph{consciously avoid allowing
  prior assumptions impact model design}, opting instead to build systems that
  learn rather than ones which encode prior hypotheses. While this is not the
  first application of deep LSTM to Bach chorales, our work consists of the
  following novel contributions.

  First, we devise \emph{a sequential encoding for polyphonic music which
  avoids hand-crafted features and resolves issues noted by prior work}. Our
  improvements include the ability to determine when notes end and a
  time-resolution at least two times greater than any prior work.

  Second, our model analysis uncovers neurons which have learned to specialize
  to detect music-theoretic concepts such as chords and cadences without any
  prior knowledge or supervision. To our knowledge, this is \emph{the first
    reported case demonstrating that LSTM is capable of learning complex
  harmonic relations automatically from music data}.

  Finally, we evaluate our automatic composition using a web-based musical
  Turing test (\url{www.bachbot.com}) on a participant pool \emph{more than three
  times larger than the next-closest comparable study} \citep{quick2014kulitta}.
  We demonstrate that a voluntary participation study promoted over social
  media can generate responses from a significant number ($165$ at time of
  writing) of highly-skilled domain experts, and is infinitely more cost efficient.
  After evaluating BachBot on $721$ participants, we found that participants
  could only differentiate BachBot's generated chorales from Bach's originals
  works only 9\% better than random guessing. In other words, \emph{generating
  stylistically successfull Bach chorales is more closed (as a result of
BachBot) than open a problem}.
\end{abstract}
