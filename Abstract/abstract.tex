% ************************** Thesis Abstract *****************************
% Use `abstract' as an option in the document class to print only the titlepage and the abstract.
\begin{abstract}
  This thesis investigates Bach's composition style using deep sequence
  learning. We develop BachBot: an automatic stylistic composition system for
  composing polyphonic music in the style of Bach's chorales.

  Our approach encodes music scores into a sequential format, reducing the task
  to one of sequence modeling. Traditional $N$-gram language models are found
  to be insufficient, prompting the use of RNN sequence models. We find a
  $3$-layer stacked LSTM performs best and conduct analyses and evaluations to
  understand its success and failure modes. Unlike many previous works, we
  avoid allowing prior assumptions about music impact model design, opting
  instead to build systems that learn rather than ones which encode prior
  hypotheses. While this is not the first application of deep LSTM to Bach
  chorales, our work consists of the following novel contributions.

  First, we devise a sequential encoding for polyphonic music which resolves
  issues noted by prior work, including: the ability to determine when notes
  end and a time-resolution exceeding all prior work by at least $2\times$.

  Second, we identify neurons which, without any prior knowledge or
  supervision, have learned to specifically detect musically meaningful
  concepts such as chords and cadences. To our knowledge, this is the first
  reported reesult demonstrating LSTM is capable of learning high-level
  musically-meaningful concepts automatically from data.

  Finally, we build a web-based musical Turing test (\url{www.bachbot.com}) and
  evaluate on a participant pool more than $3\times$ larger than the
  next-closest comparable study \citep{quick2014kulitta}. We find that a
  human evaluation study promoted over social media can yield responses from a
  significant number ($165$ at time of writing) of domain experts. After
  evaluating BachBot on $721$ participants, we found that participants could
  only differentiate BachBot's generated chorales from Bach's originals works
  \emph{only 9\% better than random guessing}. In other words, generating
  stylistically successful Bach chorales is more closed (as a result of
  BachBot) than open a problem.
\end{abstract}
