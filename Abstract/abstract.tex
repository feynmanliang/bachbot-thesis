% ************************** Thesis Abstract *****************************
% Use `abstract' as an option in the document class to print only the titlepage and the abstract.
\begin{abstract}
  This thesis investigates Bach's music composition style using deep sequence
  learning. We develop \textit{BachBot}: a deep LSTM model for automatic
  stylistic composition of polyphonic music in the style of Bach's chorales.
  Our approach encodes music scores into a sequential format, reducing the task
  to one of sequence modeling. Traditional $N$-gram language models are found
  to be insufficient, prompting the use of RNN sequence models. We find a
  $3$-layer stacked LSTM performs best and conduct analyses and evaluations to
  understand its success and failure modes.

  Unlike many previous works where model architecture is carefully designed
  using an understanding of music theory, our methodology intentionally
  minimizes prior assumptions and utilizes a vanilla LSTM without modification.
  While this is not the first application of deep LSTM to Bach chorales, our
  work consists of the following novel contributions.

  First, we devise a sequential encoding for polyphonic music which avoids
  hand-crafted features and resolves issues noted by prior work. Our
  improvements include the ability to determine when notes end and a
  time-resolution at least two times greater than any prior work.

  Second, our model analysis uncovers neurons specific to music-theoretic
  concepts such as chords and cadences. To our knowledge, this is the first
  reported case demonstrating that LSTM is capable of learning complex harmonic
  relations automatically from music data.

  Finally, we evaluate our automatic composition using a web-based musical
  Turing test (\url{www.bachbot.com}) on a participant pool larger than any
  previous study on automatic composition (to the best of our knowledge). We
  demonstrate that high-quality human evaluations, arguably superior to those
  obtained from Amazon MTurk, can be obtained through a voluntary participation
  study promoted over social media. The \todo{UPDATE} responses recieved found
  that average participants could differentiate BachBot's generated chorales
  from bach's originals works only 5\% better than random guessing.
\end{abstract}
