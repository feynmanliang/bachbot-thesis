%*******************************************************************************
%*********************************** Introduction ******************************
%*******************************************************************************
% This is the introduction where you should introduce your work.  In
% general the thing to aim for here is to describe a little bit of the
% context for your work --- why did you do it (motivation), what was the
% hoped-for outcome (aims) --- as well as trying to give a brief
% overview of what you actually did.

% It's often useful to bring forward some ``highlights'' into
% this chapter (eg some particularly compelling results, or
% a particularly interesting finding).

% It's also traditional to give an outline of the rest of the
% document, although without care this can appear formulaic
% and tedious. Your call.
\begin{savequote}[75mm]
  Since I have always preferred making plans to executing them, I have
  gravitated towards situations and systems that, once set into operation,
  could create music with little or no intervention on my part. That is to say,
  I tend towards the roles of planner and programmer, and then become an
  audience to the results.
  \qauthor{\citet{alpern1995techniques}}
\end{savequote}

\chapter{Introduction}

\ifpdf
    \graphicspath{{Chapter1/Figs/Raster/}{Chapter1/Figs/PDF/}{Chapter1/Figs/}}
\else
    \graphicspath{{Chapter1/Figs/Vector/}{Chapter1/Figs/}}
\fi

\section{Motivation}

Can the style of a particular composer or genre of music be codified into a
deterministic computable algorithm? While it may be easy to enumerate some
musical rules, reaching consensus on a formal theory for
stylistic composition has proven to be difficult. Even after hundreds of years
of study, many modern music theorists would still feel uncomfortable claiming a
``correct'' algorithm for composing music like Bach, Beethoven, or Mozart.

Despite these difficulties, recent advances in computing and progress in
modelling techniques has enabled computational modelling to provide novel
insights into various musical phenomena. By offering a method for
quantitatively testing theories, computational models can help us learn more
about the various cognitive and perceptual processes related to music
comprehension, production, and style.

One primary use case for computational music models is \emph{automatic
composition}, a task concerned with algorithmic production of musical
compositions. While early automatic composition models were predominantly
rule-based, the field has experienced an increased interest in connectionist
neural-network models over the last $25$ years. The recent empirical triumphs
of deep learning, a specific form of connectionist modelling, has further
fueled the renewed interest in connectionist systems for automatic
composition.

\section{Research aims and scope}

This thesis is concerned with \emph{automatic stylistic composition}, where the
goal is to create a system capable of generating music in a style similar to a
particular composer or genre. We restrict our attention to a particular class
of model: \emph{generative probabilistic sequence models} which are
\emph{learned from data}. A generative probabilistic model is desirable because
it can be applied to a variety of automatic composition tasks, including:
harmonizing a melody (by conditioning the model on the melody), automatic
composition (by sampling the model), and scoring (by evaluating the model on a
given sequence). Fitting the model to data enables it to automatically learn
the relationships and regularities present throughout the training data,
enabling generation of music which is statistically similar to what was
observed during training.

We develop a method for automatic stylistic composition which brings together
ideas from deep learning, language modelling, and music theory. Our motivation
stems from recent developments
\citep{srivastava2014dropout,ioffe2015batch,el1995hierarchical,schmidhuber1992learning}
which have enabled deep learning models to surpass prior state-of-the-art
techniques in domains such as computer vision, natural language processing, and
speech recognition. As it has already shown promise across a wide variety of
problem domains, we hypothesized that the application of modern deep learning
techniques to automatic composition would yield similar success.

The aim of our research is \emph{to build an automatic composition system
capable of imitating Bach's composition style on both harmonization and
automatic composition tasks in a manner that an average listener finds
indistinguishable from Bach}. While the method we develop is capable of
modelling arbitrary polyphonic music compositions, we restrict the scope of our
study to Bach's chorales. These provide a relatively large corpus by a single
composer, are well understood by music theorists, and are routinely used when
teaching music theory.

\section{Organization of the chapters}

The remaining chapters are organized as follows:

\Vref{ch:automatic-composition} describes the construction and evaluation of
our final model. Our approach first encodes music scores into a sequential
format. reducing the task to one of sequence modelling. This type of problem is
analagous to that of language modelling in speech research. Unfortuantely, we
found that traditional $N$-gram models performed poorly because they are unable
to capture the important long-range dependencies and precise harmonic rules
present in music. Inspired by the strong performance of recurrent neural
network language models, we then investigated sequence models parameterized by
recurrent neural networks and found that a deep long short-term memory
architecture performs particularly well.

In \vref{ch:model-analysis}, we open the black box and characterize the
internals of our learned model. Through measuring neuron activations to applied
stimulus, we discover that the certain neurons in the model have specialized to
specific musical concepts without any form of supervision or prior knowledge.
Our results here represent a significant milestone in computational modelling
of how musical knowledge is acquired.

We turn to the task of harmonization in \vref{ch:harmonization} and present a
method for conditionally sampling our model in order to generate
harmonizations.

To evaluate our success in achieving our stated research aim,
\vref{ch:evaluation} describes the design, results, and conclusions from a
large-scale musical Turing test we conducted. Encouragingly, we find that
average participants are only 5\% more likely than random chance to
differentiate BachBot from real Bach. Furthermore, our analysis of participant
demographics and costs suggest that voluntary participation user studies
promoted over social media yields superior data than paid studies conducted
using Amazon MTurk. This finding is especially significant to the field of
machine translation, where use of MTurk in academic publications is widely
accepted.

Finally, we summarize the conclusions from our work and suggest future directions
for extension in \vref{ch:conclusion}.
