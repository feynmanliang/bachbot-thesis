\chapter{Background}

\ifpdf
    \graphicspath{{Chapter2/Figs/Raster/}{Chapter2/Figs/PDF/}{Chapter2/Figs/}}
\else
    \graphicspath{{Chapter2/Figs/Vector/}{Chapter2/Figs/}}
\fi

The goal of this chapter is to provide a selective background in recurrent neural
networks and generative probabilistic sequence modelling required for understanding
our experiments and results. It also introduces some common definitions and
notation used throughout later chapters. We assume a basic understanding of
Western music theory. Readers unfamiliar with concepts such as piano rolls,
Roman numeral analysis, and cadences, should consult
\vref{sec:music-theory-primer} for a quick primer and
\citet{piston1978harmony,denny1960oxford} for more thorough coverage.

\section{Recurrent neural networks}\label{sec:bg-rnn}

\nomenclature[z-RNN]{RNN}{Recurrent Neural Network}

In this section, we clarify terminology and provide important background in
recurrent neural networks (RNNs). While a basic understanding of neural
networks is assumed, readers lacking such a background may be interested in
reviewing \vref{sec:primer-nn} before proceeding.

\subsection{Notation}

\nomenclature[r-input]{$\x^{(l)}_t$}{Input to layer $l$ at time $t$}
\nomenclature[r-hidden-state]{$\h^{(l)}_t$}{Hidden state of layer $l$ at time $t$}
\nomenclature[r-output]{$\y^{(l)}_t$}{Output of layer $l$ at time $t$}
\nomenclature[r-Wxh]{$\W_{st}$}{Weight matrix from source $s$ to target $t$}
\nomenclature[g-sxh]{$\sigma_{st}$}{Elementwise activation function from source $s$ to target $t$}

We first clarify the basic notation and conventions used to describe multi-layer
RNNs. Unless otherwise specified, notation appearing in the
remainder of our work is to be interpreted as defined in this section.

We use $\x_t \in \RR^{N_{in}}$ with $t=1,2,\cdots,T$ to denote a sequence of
\emph{input} (\ie observed) vectors, $\z_t \in \RR^{N_{hid}}$ a sequence of
\emph{hidden state} (\ie unobserved) vectors. and $\y_t \in \RR^{N_{out}}$ a
sequence of \emph{output} vectors.

To describe model parameters, we use $\W$ to indicate the \emph{weight matrix}
consisting of all the connection weights between two blocks of neurons
and $\sigma(\cdot)$ to indicate the \emph{activation function}. The collection
of all model parameters is denoted by $\vec{\theta}$.

When further clarity is required, we use $\W_{s,t}$ to denote the connection
weights from block $s$ to block $t$ (\ie in \cref{sec:LSTM}, $\W_{xf}$ and
$\W_{xh}$ refer to the connections from the inputs to the forget gate and
hidden state respectively). Subscripts on activation functions $\sigma_{s,t}(\cdot)$
are to be interpreted analogously.

Using the above notation, the equations for \emph{RNN time dynamics} can be expressed as
\begin{equation}\label{eq:rnn-dynamics}
 \left.\begin{aligned}
          \h_t &=& \W_{xh} \sigma_{xh} \left( \x_t \right) + \W_{hh} \sigma_{hh} \left( \h_{t-1} \right)\\
          \y_t &=& \W_{hy} \sigma_{hy} \left( \h_t \right)
       \end{aligned}
 \right\}
 \qquad \text{RNN time dynamics}
\end{equation}

When discussing multi-layer networks, we use parenthesized superscripts to
indicate layer. For example, $\z^{(2)}_t$ is the hidden states of the second
layer and $N^{(3)}_{in}$ is the dimensionality of the third layer's inputs
$\x^{(3)}_t$. We assume the outputs of the $l-1$st layer are used as the inputs
of the $l$th layer (\ie $\forall t: \x^{(l)}_t = \y^{(l-1)}_t$).

\subsection{The memory cell abstraction}

While many different variants for RNNs exist
\citep{elman1990finding,jordan1997serial,hochreiter1997long,cho2014learning,Koutnik2014,Mikolov2015},
many share the same underlying structure. Hence, it is useful to discuss RNNs
abstractly without specifying a particular variant.

\begin{figure}[tb]
  \centering
  \input{Chapter2/Figs/nn-rnn-elman.pdf_tex}
  \caption{An Elman-type RNN with a single hidden layer. The recurrent hidden
    state is illustrated as unit-delayed (denoted by $z^{-1}$) feedback edges
    from the hidden states to the input layer. The memory cell encapsulating the
  hidden state is also shown.}
  \label{fig:rnn-elman}
\end{figure}

To do so, we introduce the notion of a \emph{memory cell} in order to abstract
away how different variations of RNNs compute $\y_t$ and $\h_t$ from $\x_t$ and
$\h_{t-1}$. This is illustrated visually in \cref{fig:rnn-elman}, which shows a
standard Elman-type RNN \citep{elman1990finding} with the memory cell indicated
as a dashed box isolating the recurrent hidden state.

Notice that the edges in \cref{fig:rnn-elman} entering the memory cell
consist of the input $\x_t$ and previous hidden state $\h_{t-1}$,
and the edges leaving the memory cell consist of the current hidden state $\h_t$
and the outputs $\y_t$. Hence, to specify an concrete implementation for a
memory cell, it suffices to provide two functions $\f_h$ and $\f_y$ which
uses $\x_t$ and $\h_{t-1}$ to compute the next hidden state $\h_t = f_h(\x_t,
\h_{t-1})$ and output $\y_t = f_y(\h_t)$.

\subsection{Operations on RNNs: unrolling and stacking}

\subsubsection{Unrolling RNNs into directed acyclic graphs}

\nomenclature[z-DAG]{DAG}{Directed Acyclic Graph}

Given an input sequence $\{\x_{t}\}_{t=1}^T$ with length $T$, an RNN can be
\emph{unrolled} into a directed acyclic graph (DAG) comprised of $T$ copies of the
memory cell connected forwards in time. \Cref{fig:rnn-single-unrolled} shows
a block diagram of \cref{fig:rnn-elman} on the left and its corresponding unrolled
DAG on the right.

\begin{figure}[tb]
  \centering
  \resizebox{4.5in}{!}{\input{Chapter2/Figs/rnn-single-unrolled.pdf_tex}}
  \caption{Signal flow diagram representation of a single-layer RNN (left) and its
    corresponding DAG (right) after unrolling. The blocks labelled
    with $\h_t$ represent memory cells whose parameters are shared across all times
  $t$.}
  \label{fig:rnn-single-unrolled}
\end{figure}

\Cref{fig:rnn-single-unrolled} shows that the hidden state $\h_t$ is passed
forwards throughout the sequence of computations. This gives rise to an
alternative interpretation of the hidden state as a temporal memory mechanism.
Under this interpretation, updating the hidden state $\h_t = f_h (\x_t,
\h_{t-1})$ can be viewed as \emph{writing} information from the current inputs
$\x_t$ to memory and producing the outputs $\y_t = f_y (\h_t)$ can be
interpreted as \emph{reading} information from memory.

\subsubsection{Stacking memory cells to form deep RNNs}

In addition to unrolling, RNNs can also be \emph{stacked} to form deep RNNs
\citep{el1995hierarchical,schmidhuber1992learning}. This is accomplished in a
manner analogous to deep belief networks: outputs from the previous memory cell
in the stack are used as inputs for the current memory cell (see
\cref{fig:rnn-multi-unrolled}).

\begin{figure}[tb]
    \centering
    \resizebox{4.5in}{!}{\input{Chapter2/Figs/rnn-multi-unrolled.pdf_tex}}
    \caption{Block diagram representation of a -layer RNN (left) and its
    corresponding DAG (right) after unrolling. The blocks labelled
    with $\h_t$ represent memory cells whose parameters are shared across all times
  $t$.}
    \label{fig:rnn-multi-unrolled}
\end{figure}

Prior work has observed that ``deep RNNs outperformed the conventional, shallow RNN''
\citet{pascanu2013construct}, affirming the importance of stacking multiple layers
in RNNs. The greater modelling capabilities of multi-layer RNNs can be
attributed to three primary factors: composition of multiple non-linear
activation functions and an increase in the number of paths for information to
flow. The former reason is analogous to the case in deep belief networks, which
is well documented \citep{bengio2009learning}. To understand the latter, notice
that in \cref{fig:rnn-single-unrolled} there is only a single path from
$\x_{t-1}$ to $\y_{t}$ hence the conditional independence $\y_{t} \independent
\x_{t-1} | \h^{(1)}_t$ is satisfied. However, in \cref{fig:rnn-multi-unrolled}
there are multiple paths from $\x_{t-1}$ to $\y_{t}$ (\eg passing through
either $\h^{(2)}_{t-1} \to \h^{(2)}_t$ or $\h^{(1)}_{t-1} \to \h^{(1)}_t$)
through which information may flow.

% Additionally, parameters need not be shared
% across different layers so the stacked RNN can learn different time dynamics
% for each layer.

\subsection{Training RNNs and backpropagation through time}

\nomenclature[z-BPTT]{BPTT}{Backpropagation Through Time}
\nomenclature[g-th]{$\theta$}{Model Parameters}
\nomenclature[r-E]{$\mathcal{E}$}{Error or Loss}
\nomenclature[r-Et]{$\mathcal{E}_t$}{Error or Loss at time $t$}

The parameters $\vec{\theta}$ of a RNN are typically learned from data to
minimize some \emph{cost} $\mathcal{E} = \sum_{1 \leq t \leq T} \mathcal{E}_t(\x_t)$
measuring the performance of the network on some task. This optimization is
commonly carried out using iterative gradient descent methods, which require
computation of the gradients $\frac{\pd \mathcal{E}}{\pd \vec{\theta}}$ at each
iteration.

In feed-forward networks, computation of gradients can be performed efficiently
using backpropagation
\citep{bryson1963optimal,linnainmaa1970representation,rumelhart1988learning}.
While the cycles introduced by time-delayed recurrent hidden state connections
may seem to complicate matters for RNNs, recall that unrolling removes the
time-delayed recurrent edges and converts the RNN into a DAG (\eg
\vref{fig:rnn-single-unrolled}). The unrolled RNN can be justifiably
interpreted as a $T$ layered feed-forward neural network with parameters shared
across all layers, motivating the application of techniques such as
backpropagation to the unrolled RNNs.

Unsurprisingly, this is precisely what is done in the \emph{backpropagation
through time} (BPTT) algorithm \citep{goller1996learning}. Applying the
chain rule to the RNN dynamics equations (\vref{eq:rnn-dynamics})
unrolled network (see \cref{fig:rnn-bptt}), we obtain
\begin{align}
  \frac{\pd \mathcal{E}}{\pd \vec{\theta}} &= \sum_{1 \leq t \leq T} \frac{\pd \mathcal{E}_t}{\pd \vec{\theta}} \label{eq:err-total}\\
    \frac{\pd \mathcal{E}_t}{\pd \vec{\theta}} &= \sum_{1 \leq k \leq t} \left(
        \frac{\pd \mathcal{E}_t}{\pd \y_t}
        \frac{\pd \y_t}{\pd \h_t}
        \frac{\pd \h_t}{\pd \h_k}
        \frac{\pd \h_k}{\pd \vec{\theta}}
    \right) \label{eq:error-t}\\
    \frac{\pd \h_t}{\pd \h_k} &=
    \prod_{t \geq i > k} \frac{\pd \h_i}{\pd \h_{i-1}}
    = \prod_{t \geq i > k} \W_{hh}^\tp \diag \left( \sigma_{hh}'( \h_{i-1} ) \right)
    \label{eq:error-transfer}
\end{align}

\Cref{eq:error-t} expresses how the error $\mathcal{E}_t$ at time $t$ is a sum
of \emph{temporal contributions} $
\frac{\pd \mathcal{E}_t}{\pd \y_t}
\frac{\pd \y_t}{\pd \h_t}
\frac{\pd \h_t}{\pd \h_k}
\frac{\pd \h_k}{\pd \vec{\theta}}$
measuring how $\vec{\theta}$'s impact on $\h_k$ affects the cost
$\mathcal{E}_t$ at some future time $t > k$. The quantity
$\frac{\pd \h_t}{\pd \h_k}$ in \cref{eq:error-transfer} measures the affect of
the hidden state $\h_k$ on some future state $\h_t$ where $t > k$ and can be
interpreted as transferring the error ``in time'' from step $t$ back to step
$k$ \citep{Pascanu2012}.

\begin{figure}[tb]
    \centering
    \input{Chapter2/Figs/rnn-bptt.pdf_tex}
    \caption{The gradients passed along network edges during BPTT.}
    \label{fig:rnn-bptt}
\end{figure}

Just like traditional backpropagation, \cref{fig:rnn-bptt} demonstrates how
BPTT divides the computation of a global gradient $\frac{\pd \mathcal{E}}{\pd
\theta}$ into a series of local gradient computations, each of which involves
significantly less variables and hence is easier to compute.

\subsubsection{Vanishing/exploding gradients}

Naive implementations of RNNs (specifically \cref{eq:rnn-dynamics}) often
suffer from two well known problems: the \emph{vanishing gradient} and
\emph{exploding gradient} \citep{Bengio1994}. These problems are both related
to the product in \cref{eq:error-transfer} exponentially growing or shrinking
over long time-spans (\ie $t \gg k$). A sufficient condition (proved in
\vref{sec:vanishing-exploding-gradients}) for vanishing gradients is
\begin{equation}\label{eq:vanishing-gradients-suff}
  \left\| \W_{hh} \right\| < \frac{1}{\gamma_\sigma}
\end{equation}
where $\| \cdot \|$ is the matrix operator norm (see \vref{eq:operator-norm}),
$\W_{hh}$ is defined in \vref{eq:rnn-dynamics},
and $\gamma_\sigma$ is a constant depending on the choice of activation function
(\eg $\gamma_\sigma = 1$ for $\sigma_{hh} = \tanh$, $\gamma_\sigma = 0.25$ for
$\sigma_{hh} = \sigmoid$).

This difficulty learning relationships between events spaced far apart in time
presents a significant challenge for music applications. As noted by
\citet{cooper1963rhythmic}:
\begin{quote}
  Long-term dependencies are at the heart of what defines a style of music, with
  events spanning several notes or bars contributing to the formation of metrical and phrasal
  structure.
\end{quote}

\subsection{Long short term memory: solving the vanishing gradient}\label{sec:LSTM}

\nomenclature[z-LSTM]{LSTM}{Long Short Term Memory}
\nomenclature[z-CEC]{CEC}{Constant Error Carousel}
\nomenclature[r-input-gate]{$\i_t$}{Input gate values at time $t$}
\nomenclature[r-forget-gate]{$\f_t$}{Forget gate values at time $t$}
\nomenclature[r-output-gate]{$\o_t$}{Output gate values at time $t$}
\nomenclature[x-odot]{$\odot$}{Elementwise multiplication}

In order to build a model which learns long range dependencies, vanishing
gradients must be avoided. A popular memory cell architecture which does so is
\emph{long short term memory} (LSTM). Proposed by \citet{hochreiter1997long},
LSTM solves the vanishing gradient problem by enforcing \emph{constant error
flow} on \cref{eq:error-transfer}, that is
\begin{equation}\label{eq:const-err-flow}
    \W_{hh}^\tp \sigma_{hh}' (\h_{t}) = \matr{I}
\end{equation}
where $\matr{I}$ is the identity matrix.

As a result of the constant error flow condition, notice that \vref{eq:error-transfer}
becomes
\begin{equation}
  \frac{\pd \h_t}{\pd \h_k}
  = \prod_{t \geq i > k} \W_{hh}^\tp \diag \left( \sigma_{hh}'( \h_{i-1} ) \right)
  = \prod_{t \geq i > k} \matr{I}
  = \matr{I}
\end{equation}
The dependence on the time-interval $t-k$ is no longer present, ameliorating
the exponential decay causing vanishing gradients and enabling long-range
dependencies (\ie $t \gg k$) to be learned.

Integrating \cref{eq:const-err-flow} yields $\W_{hh} \sigma_{hh}(\h_{t}) = \h_{t}$.
Since this must hold for any hidden state $\h_{t}$, this means that:
\begin{enumerate}
    \item $\W_{hh}$ must be full rank
    \item $\sigma_{hh}$ must be linear
    \item $\W_{hh} \sigma_{hh} = \matr{I}$
\end{enumerate}

In the \emph{constant error carousel} (CEC), this is ensured by setting
$\sigma_{hh} = \W_{hh} = \I$. This may be interpreted as removing time dynamics
on $\h$ in order to permit error signals to be transferred backwards in time
(\cref{eq:error-transfer}) without modification (\ie $\forall t \geq k: \frac{\pd
\h_t}{\pd \h_k} = \I$).

In addition to using a CEC, a LSTM introduces three gates controlling access to the CEC:
\begin{description}
  \item[Input gate]: scales input $\x_t$ elementwise by $\i_t \in [0,1]$, \emph{writes} to $\h_t$
  \item[Output gate]: scales output $\y_t$ elementwise by $\o_t \in [0,1]$, \emph{reads} from $\h_t$
  \item[Forget gate]: scales previous cell value $\h_{t-1}$ by $\f_t \in [0,1]$, \emph{resets} $\h_t$
\end{description}

Mathematically, the LSTM model is defined by the following set of equations:
\begin{align}
    \i_t &= \sigmoid(\W_{xi} \x_t + \W_{yi} \y_{t-1} + \b_i) \\
    \o_t &= \sigmoid(\W_{xo} \x_t + \W_{yo} \y_{t-1} + \b_o) \\
    \f_t &= \sigmoid(\W_{xf} \x_t + \W_{yf} \y_{t-1} + \b_f) \\
    \h_t &= \f_t \odot \h_{t-1} + \i_t \odot \tanh(\W_{xh}\x_t + y_{t-1} \W_{yh} + \b_h) \\
    \y_t &= \o_t \odot \tanh(\h_t)
\end{align}
where $\odot$ denotes elementwise multiplication of vectors.

Notice that the gates ($\i_t$, $\o_t$, and $\f_t$) controlling flow in and out
of the CEC are all time varying. This can be interpreted as a mechanism
enabling LSTM to explicitly learn which error signals to trap in the CEC and
when to release them \citep{hochreiter1997long}, enabling error signals to
potentially be transported across long time lags.

\begin{figure}[tb]
    \centering
    \input{Chapter2/Figs/lstm-unit-2.pdf_tex}
    \caption{Schematic for a single LSTM memory cell. Notice how the gates $\i_t$, $\o_t$, and $\f_t$ control access to the constant error carousel (CEC).}
    \label{fig:lstm-cell}
\end{figure}

Some authors define LSTM such that $\h_t$ is not used to compute gate
activations, referring to the case where $\h_t$ is connected as ``peephole
connections'' \citep{gers2000recurrent}. We will use LSTM to refer to the
system of equations as written above.

\subsubsection{Practicalities for successful applications of LSTM}

Many applications of LSTM
\citep{devlin2014fast,zaremba2015empirical,pascanu2013construct} share some
common practical techniques for ensuring successful training. Perhaps most
important is \emph{gradient norm clipping} \citep{Mikolov2012,Pascanu2012}
where the gradient is scaled or clipped whenever it exceeds a threshold. This
is necessary because while vanishing gradients are mitigated by CECs, LSTM do
not explicitly protect against exploding gradients.

Another common practice is the use of methods for reducing overfitting and
improving generalization. In particular, \emph{dropout}
\citep{srivastava2014dropout} can be applied to the connections between memory
cells in a stacked RNN to regularize the learned features to be more robust to
noise \citep{zaremba2014recurrent}. Additionally, \emph{batch normalization}
\citep{ioffe2015batch} can also be applied to to the memory cell hidden states
to reduce co-variate shifts, accelerate training, and improve generalization.

Finally, applications of RNNs to long sequences can incur a prohibitively high
cost for a single parameter update \citep{citeulike:13881859}. For instance,
computing the gradient of an RNN on a sequence of length $1000$ costs the
equivalent of a forward and backward pass on a $1000$ layer feed-forward
network. This issue is typically addressed by only backpropagating error
signals a fixed number of timesteps back in the unrolled network, a technique
known as \emph{truncated BPTT} \citep{williams1990efficient}. As the hidden
states in the unrolled network have nevertheless been exposed to many
timesteps, learning of long range structure is still possible.

\section{Sequence probability modelling}

\mynote{Introduce this stuff better (should we assume LSTM after this point)}

In order to use LSTM as a model for music, the following assumptions
about the sequences $\x_{1:T}$, $\y_{1:T}$, and $\h_{0:T}$ are made:
\begin{enumerate}
  \item Modified Markov assumption:
    \begin{equation}
      \label{eq:modified-markov}
      \forall t: P(\h_t | \h_{0:t-1}, \x_{1:t}) = P(\h_t | \h_{t-1}, \x_t)
    \end{equation}
  \item Hidden State Stationarity:
    \begin{equation}
      \label{eq:hidden-state-stationarity}
      \forall t_1, t_2: P(\h_{t_1} = \k | \h_{t_{1}-1} = \i, \x_{t_1} = \j) = P(\h_{t_2} = \k | \h_{t_{2}-1} = \i, \x_{t_2} = \j)
   \end{equation}
  \item Output Stationarity:
    \begin{equation}
      \label{eq:output-stationarity}
      \forall t_1, t_2: P(\y_{t_1} = \j | \h_{t_1} = \i) = P(\y_{t_2} = \j | \h_{t_2} = \i)
   \end{equation}
  \item Output independence:
   \begin{equation}
     \label{eq:output-independence}
     P(\y_{1:T} | \h_{0:T}, \x_{1:T}) = \prod_{t=1}^T P(\y_t | \h_{t}, \x_t)
   \end{equation}
\end{enumerate}

These assumptions imply the sequential factorization:
\begin{align}
  &P(\y_{1:T}, \h_{1:T} | \h_0, \x_{1:T})  &\\
  &= P(\y_{1:T} | \h_{0:T}, \x_{1:T}) P(\h_{1:T} | \h_0, \x_{1:T})  &\\
  &= \left( \prod_{t=1}^T P(\y_t | \h_t) \right) P(\h_{1:T}| \h_0, \x_{1:T}) &\text{\cref{eq:output-independence}}\\
  &= \left( \prod_{t=1}^T P(\y_t | \h_t) \right) \left(\prod_{t=1}^T P(\h_{t}| \h_{0:t-1}, \x_{1:t})\right) &\\
  &= \left( \prod_{t=1}^T P(\y_t | \h_t) \right) \left(\prod_{t=1}^T P(\h_{t}| \h_{t-1}, \x_{t})\right) &\text{\cref{eq:modified-markov}}\\
  &= \prod_{t=1}^T P(\y_t | \h_t, \x_t) P(\h_{t}| \h_{t-1}, \x_{t}) & \\
\end{align}
Together, \cref{eq:hidden-state-stationarity} and \cref{eq:output-stationarity} imply that $P(\y_t | \h_t)$
and $P(\h_{t}| \h_{t-1}, \x_{t})$ are time-invariant and can be modelled by the same recurrent function.

In RNNs, the hidden state dynamics $P(\h_t | \h_{t-1}, \x_t)$ are deterministic:
\begin{equation}
  \h_t = f_h(\x_t, \h_{t-1})
\end{equation}
Which means that $P(\y_{1:T}, \h_{1:T} | \h_0, \x_{1:T}) = P(\y_{1:T} | \h_0, \x_{1:T})$.
This yields the factorization
\begin{equation}
  P(\y_{1:T} | \h_0, \x_{1:T})
  = P(\y_{1:T}, \h_{1:T} | \h_0, \x_{1:T})
  = \prod_{t=1}^T P(\y_t | \h_t, \x_t) f_h(\x_t, \h_{t-1})
\end{equation}
\mynote{Draw PGM}

However, one minor problem remains. Let $\z_t = f_y(f_h(\x_t, \h_{t-1}))$ (with
$f_y$ and $f_h$ as defined in \mynote{ref}) denote the outputs of the RNN model
at time $t$. Note that $\z_t$ can be any real vector in $\RR^{|V|}$
\mynote{Define $V$ to be the vocabulary}, but $P(\x_{t+1} | \h_{t-1}, \x_{t})$ is
a probability vector constrained to sum to one.

Fortunately, we can treat $\z_t$ as the \emph{scores} for a \emph{Boltzmann
distribution}
\begin{equation}\label{eq:boltzmann-dist}
    P( \y_{t} = s | \h_{t-1}, \x_t )
    = \frac{\exp \left(-\z_{t,s}/T\right) }{ \sum_{k=1}^{K} \left(\exp -\z_{t,k}/T\right)}
\end{equation}
where $T \in \RR^+$ is a \emph{temperature} parameter (set to $T=1$ during training and varied during sampling).
To keep notation compact, we omit writing this explicitly and understand $P(\y_t | \h_{t-1}, \x_t)$ to mean
the Boltzmann distribution parameterized by the scores $f_y(f_h(\x_t, \h_{t-1}))$.

Note the similarity between \cref{eq:modified-markov}--\cref{eq:output-independence}
and the assumptions for Hidden Markov models \citep{ramage2007hidden}. Discrepancies are due
to the presence of an input sequence $\x_{1:T}$ in our sequence-to-sequence model.

\mynote{Discuss validity of assumptions, namely output independence assuming hidden state and input summarize
all prior context}


