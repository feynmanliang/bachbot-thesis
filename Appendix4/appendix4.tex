% ******************************* Thesis Appendix D ****************************
\chapter{Appendix D: Description of Software}\label{sec:software}

\ifpdf
    \graphicspath{{Appendix4/Figs/Raster/}{Appendix4/Figs/PDF/}{Appendix4/Figs/}}
\else
    \graphicspath{{Appendix4/Figs/Vector/}{Appendix4/Figs/}}
\fi

This appendix chapter provides a listing of the key
elements of software as well instructions for accessing
the code and reproducing the experiments. The software
is divided into three different repositories:
\texttt{bachbot} includes tools to preprocess musical
data and train/sample/harmonize/score LSTM sequence
models, and the other two repositories are the
front-end javascript client and back-end
\texttt{node.js}/Azure server.

\section{\texttt{bachbot}}

The \texttt{bachbot} repository contains code related to:
\begin{enumerate}
    \item Preprocessing and sequential encoding for polyphonic music (\path{bachbot/scripts/datasets.py})
    \item A modified version of \texttt{torch-rnn} supporting harmonization and
        UTF8 token sequences (\path{bachbot/scripts/harm_model})
    \item Training \texttt{torch-rnn} the \texttt{bachbot} sequence model (\path{bachbot/scripts/train.py})
    \item Automatic composition (\ie sampling) with a trained model (\path{bachbot/scripts/sample.py})
    \item Harmonization with a trained model (\path{bachbot/scripts/harm_model/harmonize.lua})
    \item Benchmarking against $N$-gram language models \path{bachbot/scripts/benchmarks/}
    \item Benchmarking against other memory-cell implementations \path{bachbot/scripts/theanet/}
\end{enumerate}

We provide the following files to demonstrate some use cases:
\begin{enumerate}
    \item \path{bachbot/scripts/0-prepare_all.zsh} -- prepares preprocessed and encoded corpuses
        for model training, automatic composition, and harmonization
    \item \path{bachbot/scripts/1-train.zsh} -- trains the LSTM sequence model with parameters used
        for our experiments
    \item \path{bachbot/scripts/2-sample-decode.zsh} -- automatic composition; samples a token
        sequence from the LSTM and decodes into \texttt{musicxml}
    \item \path{bachbot/scripts/3-harmonize.zsh} -- harmonization; performs all harmonization tasks,
        decodes and scores the results
\end{enumerate}

\section{Subjective evaluation}

We provide our infrastructure for conducting large-scale web-based human evaluation, which can be
easily adapted to other applications. Our code is split into two parts: front-end and back-end.

\subsection{\texttt{subjective-evaluation-client}}

The client is written in Javascript (ECMAScript 2016) and requires compilation
(\texttt{npm run build}) before it can be deployed. Some important
files/folders:
\begin{enumerate}
    \item \path{src/components} -- contains the React code for front-end
        components (\eg audio playback, quiz question-response form, user-info
        form, landing page)
    \item \path{src/redux} -- contains the Redux code for application state management
        and data collection
    \item \path{test} -- contains tests documenting and enforcing correct application behavior
\end{enumerate}

\subsection{\texttt{subjective-evaluation-server}}

The server is written using \texttt{node.js} and requires an Azure connection string to be
available in the shell environment. The repository is organized as follows:
\begin{enumerate}
    \item \path{src/app.js} -- static serving of \texttt{experiments.json} and handling
        of POST \texttt{/submitResponse} which persists responses to Azure BlobStorage
    \item \path{src/public} -- directory for static assets
    \item \path{scripts/} -- utilities for interacting with Azure (\eg setting CORS headers,
        uploading experiments, downloading responses)
\end{enumerate}

\section{Instructions for access and reproducting experiments}

All software has been made open-source and is available on GitHub:
\begin{itemize}
    \item \url{https://github.com/feynmanliang/bachbot}
    \item \url{https://github.com/feynmanliang/subjective-evaluation-client}
    \item \url{https://github.com/feynmanliang/subjective-evaluation-server}
\end{itemize}

To reproduce experimental results, we recommend cloning the \texttt{bachbot} repository
and exploring \path{README.md} and the source code in \path{scripts/bachbot.py}.

Although we use Azure for the back-end, \texttt{subjective-evaluation-client}
was written to be vendor-agnostic and can be delivered by any static content
web-server or content delivery network. Its dependencies are:
\begin{itemize}
    \item A URL providing \texttt{experiments.json}
    \item A REST end-point handling POSTs to \texttt{/submitResponse} which persists
        JSON blobs containing user responses
\end{itemize}
After cloning the repository, run \texttt{npm install \&\& npm start}.

A Microsoft Azure account is required for deploying \texttt{subjective-evaluation-server}.
Its dependencies are:
\begin{itemize}
    \item Azure App Service -- for running the \texttt{node.js} web server
    \item Azure BlobStorage -- for persisting JSON blobs of user responses
\end{itemize}
After cloning the repository, run \texttt{npm install \&\& npm start} to start the server
locally. Look at \texttt{npm run deploy} for more details on deploying to Azure.

Although not required, we encourage the use of a CDN to serve \texttt{mp3}
files, parameters for the current quiz questions (\texttt{experiments.json}),
and even the compiled front-end javascript (\texttt{bundle.js}). We use Azure
CDN for this purpose.

IPython notebooks reproducing the data analyses, figures, and tables shown in this
dissertation are available at
\url{https://github.com/feynmanliang/bachbot-thesis}.
