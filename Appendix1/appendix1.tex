% ******************************* Thesis Appendix A ****************************
\chapter{Appendix A}

\section{Vanishing/exploding gradients}\label{sec:vanishing-exploding-gradients}

Following \citet{Pascanu2012}, let $\| \cdot \|$ be any
submultiplicative matrix norm (\eg Frobenius, spectral, nuclear, Shatten
$p$-norms). Without loss of generality, we will use the \emph{operator norm}
defined as
\begin{equation}
    \| A \| = \sup_{x \in \RR^n; x \neq 0} \frac{|A x|}{|x|}
\end{equation}
where $|\cdot|$ is the standard Euclidian norm.

From submultiplicativity, we have that for any $k$
\begin{equation}
    \left\| \frac{\pd \h_k}{\pd \h_{k-1}} \right\|
    \leq \| \W_{hh}^\tp \| \| \diag\left( \sigma_{hh}'(\h_{k-1}) \right) \|
    \leq \gamma_{\W} \gamma_\sigma
\end{equation}
where we have defined $\gamma_{\W} = \| \W_{hh}^\tp \|$ and
\begin{align}
    \gamma_\sigma
    &\coloneqq \sup_{h \in \RR^n} \| \diag \left( \sigma_{hh}'(\h) \right) \|  &\\
    &= \sup_{h \in \RR^n} \max_i \sigma_{hh}'(\h)_i &\mbox{Operator norm of diag} \\
    &= \sup_{x \in \RR} \sigma_{hh}'(x) &\mbox{$\sigma_{hh}$ acts elementwise}
\end{align}

Substituting back into \cref{eq:error-transfer}, we find that
\begin{equation}
    \left\| \frac{\pd \h_t}{\pd \h_k} \right\|
    = \left\| \prod_{t \geq i > k} \frac{\pd \h_i}{\pd \h_{i-1}} \right\|
    \leq  \prod_{t \geq i > k} \left\| \frac{\pd \h_i}{\pd \h_{i-1}} \right\|
    \leq (\gamma_{\W} \gamma_\sigma)^{t-k}
\end{equation}

Hence, we see that a sufficient condition for vanishing gradients is
for $\gamma_{\W} \gamma_\sigma < 1$, in which case $\left\| \frac{\pd \h_t}{\pd \h_k} \right\| \to 0$
exponentially for long timespans $t \gg k$. %$\qed$

If $\gamma_\sigma$ is bounded, sufficient
conditions for vanishing gradients to occur may be written as
\begin{equation}
    \gamma_{\W} < \frac{1}{\gamma_\sigma}
    \label{eq:vanishing-gradients-suff}
\end{equation}
This is true for commonly used activation functions (\eg $\gamma_\sigma = 1$
for $\sigma_{hh} = \tanh$, $\gamma_\sigma = 0.25$ for $\sigma_{hh} =
\sigmoid$).

The converse of the proof implies that $\| \W_{hh}^\tp \| \geq
\frac{1}{\gamma_\sigma}$ are necessary conditions for $\gamma_{\W}
\gamma_\sigma > 1$ and exploding gradients to occur.
